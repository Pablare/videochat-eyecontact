\documentclass[main.tex]{subfiles}

\begin{document}
	\section{Expose}
	Der Sichtkontakt zwischen zwei Personen bei der Videotelefonie ist nicht gegeben,
	da die Punkte der Kameralinse und Blickpunkt der jeweiligen Person naturgegeben von einander entfernt
	sind,wenn der Blick der Person auf den Bildschirm gerichtet ist.
	Ziel meiner Seminararbeit ist es Verfahren zu entwickeln und zu Bewerten die dieses Problem lösen.
	Dazu gibt es bereits einige Ansätze mit dessen Ergebnis ich den Vergleich suchen werde.
	Im Rahmen meiner Seminararbeit werde ich eine Software entwickeln die,
	an Hand der nach dem Stereoskopieverfahren aus zwei Kamerabildern gewonnenen Tiefeninformationen und Bildinformationen, ein Bild berechnet, 
	welches das gleiche Objekt zeigt wie die beiden Quellenbilder,
	dessen Blickwinkel sich jedoch von denen der beiden Quellenbilder unterscheidet.
	Die Arbeit enthält Erklärungen der verwendeten Algorithmen und mathematischen Prinzipien. 
	Hauptsächlich wird das Prinzip der Korrelation von Bedeutung sein.
	Die Korrelation ist der Zusammenhang zwischen zwei Zufallsgrößen, in dieser Anwendung
	ein Mittel zur Bestimmung der Ähnlichkeit von zwei Bildpunkten.
	Es wird untersucht wie gut sich welche Algorithmen sich für das Verfahren eignen. 
	Zu diesem Zweck wird ein Verfahren zur Bewertung der der Qualität des Ergebnisses entwickelt werden.
\end{document}